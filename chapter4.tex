%!TEX root = board-prep.parent.tex

\section{基本事項}
\begin{itemize}
\item 気管支ビデオスコープの先端には\textbf{CCD (charged coupled device)}が装着される.CCDの小型化には限界があるため,先端に光ファイバーを装着し,スコープ内にCCDを備え付ける\textbf{ハイブリッドスコープ}も

\end{itemize}

\section{気管支鏡の観察項目}
\subsection{所見}
\begin{itemize}

\item 内視鏡的層別分類は,内側から\textbf{上皮層}(上皮+基底膜),\textbf{上皮下層}(上皮化血管,縦走襞),\textbf{壁内層}(平滑筋,軟骨),\textbf{壁外層}の4つである.

\item 縦走襞は\textbf{上皮下層の深部の弾力線維}であり,輪状襞は\textbf{壁内層の平滑筋}である.

\item 気管支表面の白色調の変化は\textbf{萎縮性気管支炎,腫瘍の上皮直下への進展,貧血}で認める.
\item 気管支内腔における黒褐色の色調変化は\textbf{上皮下}の炭粉沈着を反映する.
\item 気道狭窄のうち,外からの圧排+内腔内にも所見を認めるものは\textbf{混合性狭窄}である.
\item 気道狭窄のうち,ステント治療の適応から外れるのは\textbf{腔内性狭窄}である.
\item 機能性狭窄は\textbf{気管・気管支軟化症},\textbf{excessive dynamic airway collapse (EDAC)}などの支持組織の異常による.

\item 隆起性変化のうち,$>$2mmの高さを\textbf{結節}と呼ぶ.$<$2mmの変化は\textbf{平坦性変化}と呼ぶ.
\item 平坦性変化のうち,「肥厚」は\textbf{気管支分岐部}における所見として用いる.\textbf{Sq}が代表的である.
\item 平坦性変化のうち,気管支表面の凹凸不整な変化を\textbf{顆粒状(凹凸不整)}とよぶ.肺癌などの悪性疾患のほか,\textbf{サルコイドーシス}でも認める.
\item 平坦性変化のうち,上皮組織の欠損による\textbf{陥凹性変化}は,\textbf{気管支結核},\textbf{Sq}の一部で認める.

\item 上皮下血管の増生は,\textbf{肺癌のリンパ節転移},\textbf{縦隔腫瘍},\textbf{サルコイドーシス},\textbf{SVC症候群}などでみられる.
\item 軟骨・軟骨輪の増生性変化は,\textbf{気管気管支骨軟骨形成症}で認める.

\end{itemize}
\subsection{腫瘍の深達度診断}
\begin{itemize}


\item \textbf{上皮下血管の消失}がある場合,腫瘍は\textbf{上皮層}に存在する.
\item 腫瘍の\textbf{表面の色調変化}がある場合,少なくとも腫瘍は\textbf{上皮下層}以浅にある.特に,上皮下浅層まで及べば縦走襞が\textbf{消失}する.
\item 腫瘍が壁内層に広く浸潤する場合,\textbf{輪状襞}は消失し,場合によって\textbf{縦走襞の圧縮強調}がみられる.\textbf{Ad, Sm}に典型的であり,切除範囲は慎重に決定する!

\item bridging foldは,\textbf{壁外層}または\textbf{壁内層}から圧排を受けることで,縦走襞が盛り上がって見える所見である.bridging foldの途中で縦走襞が断裂していれば,\textbf{上皮浅層}までの進展を意味する.

\item 壁内層または壁外層の腫瘍が気管・気管支壁に浸潤しているか判断したい場合,\textbf{EBUS}を使う.上皮を主体とした病変で境界がわかりにくい場合,\textbf{AFI}や\textbf{NBI}を用いる.


\end{itemize}

\section{処置と検体処理}
\subsection{R-EBUS}

\begin{itemize}
\item R-EBUSを用いて\textbf{中枢側}の気管支壁の深達度診断を行う場合,プローブをバルーンシースで膨らませて空気を抜き,超音波画像を得る.
\item 膜様部は3層構造,そのほかの肺外気管支と肺内気管支は5層構造.
\item 肺外気管支の5層構造は,内側から境界エコー(high),上皮下組織(low),気管支軟骨の内側縁(high),気管支軟骨(low),気管支軟骨の外側縁(high).気管支軟骨より浅い病変でPDTが有効なので,\textbf{第4層(気管支軟骨のローエコー)}を拾うことが大切.
\item EBUS-GSによる生検回数は\textbf{5-6回}でピークに近づく.
\end{itemize}

\subsection{EBUS-TBNA}
\begin{itemize}
\item TBNAスコープを食道に挿入することも可能で,EUS-B-FNAと呼ぶ.\textbf{\#8, \#9}はEUS-B-FNAの絶対適応で.\textbf{\#5, \#7}の一部もEUS-B-FNAでしか生検できないことがある.
\item 腫瘍のリンパ節ステージングでは,\textbf{3 station}以上の生検が必須で,その中に\textbf{\#4R, \#4L, \#7}を含める.
\item リンパ節の左右の血管は,\#2Rが\textbf{SVC-BCA},\#4Rが\textbf{Az-SVC},\#4Lが\textbf{PA-Ao}.
\item EBUS-TBNAの穿刺において\textbf{陰圧は必須ではない.}ROSEをしない場合,同一部位に対して最低\textbf{3回}は穿刺する.
\item 転移のないリンパ節は,形状は楕円形,辺縁は\textbf{不明瞭}で,内部は中心部の\textbf{高エコー}を伴い,内部血流は\textbf{少なく},\textbf{リンパ節門の血流}が観察される.逆に,転移性のリンパ節は\textbf{気管支動脈}からの流入血管を認める.
\end{itemize}

\subsection{AFB}

\begin{itemize}

\item AFBは普通のBFSでは気管支壁に紛れて見逃してしまうような,\textbf{早期癌病変}を見つけるモダリティである.

\item \textbf{青色}(420-460nm)の励起光を気管支の正常部に照射すると,\textbf{緑色}(480-520nm)の自家蛍光が発生するが,腫瘍部ではこの波長の自家蛍光が減弱するので,正常部とのコントラストを増幅して観察できる.
\item AFBの偽陽性となるのは\textbf{慢性炎症}や\textbf{血管の増生}である.
\end{itemize}

\subsection{NBI}
\begin{itemize}
\item NBIは\textbf{毛細血管を強調}することで,特に\textbf{Sq}のような多段階発癌の様子を推定できる.
\item 気管支の正常部は\textbf{青色}と\textbf{緑色}の光を反射するが,逆にHbはこれらの色を吸収する(特に青色を強く吸収する).
\item \text{上皮〜上皮下表層}の血管像が\textbf{茶色}で,表層下のやや深い血管が\textbf{緑色〜シアン色}に見える.
\item 気管支Sqは,\textbf{squamous dysplasia},\textbf{ASD (angiogenic squamous dysplasia)},\textbf{CIS (carcinoma in situ,上皮内癌)},微小浸潤癌,浸潤癌と進展していく.
\item NBIにおいて,微細な血管網の増生,蛇行がみられたら,\textbf{squamous dysplasia}または\textbf{ASD}を考える.点状血管があれば\textbf{ASD}以上で,いよいよらせん型・スクリュー型の血管がみられると\textbf{CIS}以上と考える.



\end{itemize}

\subsection{その他のモダリティ}

\begin{itemize}

\item 超細径気管支鏡は外径\textbf{3.5mm}以下で,処置チャンネルは\textbf{1.7mm}が主流となっている.基本は経鼻挿管で,\textbf{カフなしφ5mm}のチューブを先に挿入しておく.

\item \textbf{EMN (electromagnetic navigation)}は,位置センサーを内蔵したガイドシースをスコープに通した状態で,リアルタイムなスコープの先端位置を再構成画像上に描出する.位置センサーが目標位置に到達したらセンサーを抜いてガイドシースを残し,あとは普通の生検と同じ.
\item \textbf{CLE (confocal laser endomicroscopy)}は,末梢型肺腫瘍において,ROSEと同じような細胞観察を,BFS越しに行う.細径シースにミニプローブ(φ1.4mm)を挿入して腫瘍まで誘導したのち,\textbf{フルオレセイン}を注射して観察する.
\item OCTは\textbf{上皮層}と\textbf{上皮下層}までは観察できるので,末梢のCISの診断などに応用可能.
\end{itemize}

\subsection{検体採取と処理}
\begin{itemize}

\item TBB/TBLBは検体採取終了後,2-3分はwedgeし,検査2時間後にXp撮影.
\item キュレットは中枢気道病変でスコープのアングルを目一杯使っても到達できない部位の擦過も可能である.
\item 気管支洗浄を行うときは\textbf{リドカイン}を控える(殺菌作用があるため).
\item クライオプローブの外径はφ2.4mmとφ1.9mmが一般的.末梢肺病変の生検を行う場合,φ1.1mmのプローブで気管支鏡を抜かずに生検する(miniCryo)も有望.
\item 細胞診検体(特にTBNA検体)の場合,血液成分が多いので,\textbf{スライドガラスを擦り合わせて}検体を伸ばす.ただし組織の挫滅に注意する.
\item 普通の細胞診は\textbf{95\%エタノール}を用いて湿固定し,Papanicolaou染色を行う.\textbf{乾燥}を避ける.
\item 細胞診検体を遺伝子検査に使う場合,擦過器具を\textbf{生食}または\textbf{リン酸緩衝食塩液(PBS)}で洗ったものを保管する.
\item ROSEを行う場合,よく乾燥させて,Diff-QuikやCytoQuickなどの簡易法で染色する.May-Giemsa染色と同様の見えになる.
\item DNA,RNAの解析には,基本的には\textbf{凍結保存(液体窒素)}がbest.遺伝子解析を行うだけなら\textbf{10\%中性緩衝ホルマリン溶液}で\textbf{6-48時間}固定して,FFPE(ホルマリン固定パラフィン包埋)でもOK.ホルマリン固定を長時間行うとDNAの断片,修飾が起こる.
\item PD-L1はFFPE薄切後\textbf{6週}以内に検査する.
\item 生検体での検査が必要なのは\textbf{FCM},\textbf{染色体分析},\textbf{T細胞/B細胞受容体遺伝子の遺伝子解析}.
\item ホルマリン固定の検体でOKなのは,\textbf{FISH}や\textbf{PCR},\textbf{NGS}.



\end{itemize}
\newpage


\section{疾患}
\subsection{喉頭〜気管腫瘍}
\begin{itemize}

\item 乳頭腫は\textbf{喉頭}または\textbf{気管〜気管支}に発症する(HPV6/11と関連).気管乳頭腫は\textbf{扁平上皮性乳頭腫},\textbf{腺上皮性乳頭腫},\textbf{混合型}の3つに分かれる.癌化するのは\textbf{扁平上皮性乳頭腫}.


\item 喉頭癌の前癌病変として\textbf{喉頭白板症}があり,CISとの鑑別は\textbf{声帯の可動性}や\textbf{異常血管}の有無に注目する.
\item 喉頭癌の頻度は\textbf{声門},\textbf{声門上},\textbf{声門下}の順.
\item 声帯ポリープは普通のポリープ,ポリープ様声帯は声帯全体が浮腫む.

\item 気管腫瘍で上皮下血管が拡張 ▶ \textbf{平滑筋腫},\textbf{軟骨腫},\textbf{カルチノイド},\textbf{多形腺腫}.
\item 平滑筋腫のマーカーは\textbf{α-SMA},\textbf{カルデスモン},筋原性のマーカーは\textbf{デスミン},\textbf{HHF35}.
\item 気管腫瘍で有茎状 ▶ \textbf{脂肪腫}.
\item 気管〜気管支腫瘍でS-100蛋白陽性 ▶ \textbf{顆粒細胞腫}または\textbf{神経原性腫瘍}.
\item 顆粒細胞腫はその名の通りBFS所見は\textbf{凹凸不整(顆粒状)}.悪性転化したり重複癌のメルクマールになる.
\item 気管腫瘍で「硬く生検困難」▶ \textbf{軟骨腫},\textbf{脂肪腫}(表面の被膜が硬いので).

\item 腺様嚢胞癌は組織学的には\textbf{唾液腺型腫瘍}であり,\textbf{管腔型}より\textbf{充実型}で予後不良.

\item 気管支腫瘍は\textbf{良性と分かっていれば内視鏡的治療(YAG,スネア)}.逆に外科的治療の条件は,①悪性の可能性あり,②腫瘍による気道閉塞が高度,③気管支壁への腫瘍浸潤.
\end{itemize}
\subsection{気管支〜肺の悪性腫瘍}

\begin{itemize}

\item ポリープ型の悪性腫瘍の鑑別として\textbf{早期Sq},\textbf{大細胞癌},\textbf{多形腺腫},\textbf{平滑筋肉腫}.

\item ★内視鏡的早期肺癌の基準:\textbf{気管から亜区域枝}\footnote{つまりB1aまで.}までに限局,\textbf{腫瘍の末梢辺縁が内視鏡的に可視},\textbf{長径$<$2cm},\textbf{組織学的にSq}.
\item 内視鏡的早期肺癌は\textbf{平坦型},\textbf{結節型},\textbf{早期ポリープ型}に分類される.平坦型においてCISの可能性が高いのは\textbf{浸潤範囲$<$10mm}.
\item 進行Sqは\textbf{粘膜型},\textbf{粘膜下型},\textbf{壁外型}に分類される.
\item SCLCは\textbf{上皮下型}または\textbf{壁内型}で多彩な病像を呈するが,前者であれば上皮化血管の消失,後者であれば縦走襞の圧縮強調を伴う.


\item 甲状腺癌の気管浸潤は\textbf{気管壁に接して易出血性の発赤}を認める.食道癌の気管浸潤はどちらかというと\textbf{亜全周性}の病変になることが多い.

\item 肺転移の鑑別で,CK7+CK20+なら\textbf{尿路上皮癌,膵癌},CK7-CK20+なら\textbf{大腸癌},CK7-CK20-なら\textbf{肝細胞癌,前立腺癌,腎細胞癌}.
\item メラノーマの気管転移は\textbf{気管支内腔のポリープ状の発育}で,表面は\textbf{青みを帯びた黒色調}のことが多い(ただし転移病変がメラニンを含まないことはよくある).

\item 肺癌以外の気管支悪性腫瘍で,完全切除が可能なのは\textbf{平滑筋肉腫}と\textbf{多形腺腫}.\textbf{多形腺腫}はBFSで治療が完結できる!
\item 出血しやすい悪性腫瘍 ▶ \textbf{腎癌の気管支転移}.
\end{itemize}




\subsection{結核}
\begin{itemize}

\item 喉頭結核で最も多いのは\textbf{声帯$>$喉頭蓋$>$仮声帯}で,肉芽腫型が多い.
\item 気管・気管支結核の定義は,「\textbf{区域気管支}\footnote{つまりB1まで.}おより中枢側の気管・気管支の結核性粘膜病変」で,\textbf{左主気管支}に好発する.
\item 気管支結核の内視鏡分類(荒井分類):早期病変は\textbf{Type II(粘膜内結節型)}である.白苔を伴う潰瘍形成がある場合は\textbf{Type III(潰瘍型)}で,隆起性潰瘍であれば\textbf{Type IIIb}.
\item 結核の治癒過程に入ると\textbf{Type IV(肉芽型)}になり,最終的に\textbf{Type V(瘢痕型)}になる.狭窄を伴う場合は\textbf{Type Vb}.

\item 結核性リンパ節炎が気管支内に穿破すると\textbf{Type LN}で,古典的には小児・若年成人に多い.

\end{itemize}


\subsection{真菌感染症}

\begin{itemize}

\item ABPMでの病理では\textbf{Charcot-Leyden結晶}(好酸球性炎症を反映するタンパク質が結晶化)が特徴的.
\item PCPのBALFでは\textbf{卵円形嚢子}がクラスターで観察されれば確定.
\end{itemize}

\subsection{アレルギー・間質性肺疾患}
\begin{itemize}

\item 喘息患者のうち,好中球性の気道炎症が強い場合は\textbf{Th17}が関与する.ステロイド抵抗性の重症喘息では\textbf{Th2/Th17 positive}の細胞増生がある.
\item 喘息患者のBALFではLT,PG,TX,ヒスタミン,\textbf{トリプターゼ}などのメディエーターに加えて,\textbf{TGF-β1}や\textbf{bFGF}の増加がみられる.
\item HPでは基本的にCD4/CD8比は\textbf{上昇}(夏型を除く).CHPでも\textbf{上昇}.
\item 他にCD4/CD8比の上昇は\textbf{MTX薬剤性肺炎},\textbf{サルコイドーシス},\textbf{慢性ベリリウム肺}.

\item 血管炎のうちGPAはBFSの可視範囲において発赤,腫脹を広範囲に認める.\textbf{声門下狭窄}も17\% に認める.
\item 血管炎のBALFは,MPAは特徴的な所見はなく,GPAは\textbf{Ny⧺,Eo+},EGPAでは\textbf{Eo⧺}.
\item 名前の通り,GPAの病理では\textbf{肉芽腫→壊死},\textbf{血管炎}がすべて揃う.MPAの病理は基本的に肉芽腫はない.

\end{itemize}

\subsection{サルコイドーシス}

\begin{itemize}

\item サルコイドーシスのBFS所見:\textbf{上皮下血管の増生}(いわゆるnetwork formation)と,上皮組織の混濁による\textbf{上皮化血管の消失}のどっちもあり!内腔の\textbf{顆粒状変化}は若年層に多い.
\item サルコイドーシスのI期は\textbf{BHL},II期は\textbf{BHL+肺陰影},III期は\textbf{肺陰影 alone},IV期は\textbf{肺線維化}(LN→肺実質と炎症が移行する).\end{itemize}

\subsection{アミロイドーシス}

\begin{itemize}

\item アミロイドーシスのBFS所見では,\textbf{プラーク}または\textbf{隆起性病変}が飛び飛びに見られることに加えて,\textbf{血管増生}が目立ち,大出血することもある!
\item 気管・気管支アミロイドーシスでは\textbf{AL}が多く,全身性アミロイドーシスの気道病変として現れる場合は\textbf{AA}が多い.進行例では気管支狭窄が起こり得る.
\item 肺野結節型のアミロイドーシスはAA, ALのいずれでも起こる.
\item びまん性AL型アミロイドーシスの肺病変ととして\textbf{びまん性肺胞隔壁アミロイドーシス}があり,小葉間隔壁の肥厚などIP likeな画像所見を呈する.
\end{itemize}

\subsection{稀な疾患}

\begin{itemize}
\item LCHは進行すると組織中のLangerhans細胞は消失する.細気管支周辺に\textbf{stellate fibrotic scar}を認める.BALFではCD4/CD8は\textbf{低下}.



\item 石綿肺のBALFにおいて肺癌発症リスクを高める所見として\textbf{BALF 1mLあたり石綿小体が5本以上}がある.
\item 珪肺のBALFでは\textbf{Ly+}を認め,CD4/CD8比は\textbf{低下}する.
\item 溶接工肺は,BALFまたはTBLBで\textbf{鉄染色陽性のMph}を認める.

\item PAPのBALFでは,\textbf{顆粒状の無構造物質}の沈着と,\textbf{泡沫状Mph}がみられ,これらはPapanicolaou染色で\textbf{ライトグリーン}に染まる.
\item PAPの病理所見では,末梢気腔内に\textbf{弱好酸球性細顆粒状物質}を認め,これは\textbf{PAS}染色で陽性,\textbf{SP-A}にも陽性である.随伴所見として,末梢気腔内に\textbf{大型泡沫細胞}が集積する.

\item LAMのBALFでは\textbf{ヘモジデリン貪食像を伴うMph}の増加を認める.LAM細胞は平滑筋の性質を持つので\textbf{HMB-45}が陽性であれば確定診断(ただしLAM細胞は集塊で認めるので,すべてのTBLB検体で見られるわけではない).LAM細胞は\textbf{PgR}が陽性のものもある.
\item LAMでは血清で\textbf{VEGF-D}の上昇が診断に有用(保険未収載).

\item 気管支結石は,古典的には\textbf{結核}や\textbf{ヒストプラズマ}との関連が重要.\textbf{管内性}(気管支内部で発育)または\textbf{管外性}(気管支周囲リンパ節の気管支内穿孔).
\end{itemize}



\subsection{気管・気管支骨軟骨形成症(TO)}

\begin{itemize}

\item 気管・気管支骨軟骨形成症(TO)は,気道の\textbf{粘膜下組織}に,骨・軟骨組織が異所性に増生する.
\item 好発部位は\textbf{気管近位部〜中間部}で,\textbf{主気管支}くらいまで病変がある.ただし,\textbf{膜様部}には病変はない.
\item TOの組織所見は\textbf{扁平上皮化生}と,骨・軟骨,石灰化,\textbf{骨化領域の造血骨髄}をみる.
\item CTでも多発石灰化病変認めるが,鑑別診断は\textbf{再発性多発軟骨炎},\textbf{サルコイドーシス},\textbf{アミロイドーシス},\textbf{乳頭腫}などが主な鑑別となる.


\end{itemize}

\subsection{気管・気管支軟化症}

\begin{itemize}

\item 気管・気管支軟化症(TBM)は高度の咳発作の反復によっても起こるが,secondaryに起こるものとして\textbf{軟骨損傷},\textbf{気管・気管支結核},\textbf{再発性多発軟骨炎},\textbf{Mounier-Kuhn症候群}\footnote{気管気管支の内腔が異常に拡張する稀な疾患で,しばしば慢性呼吸器感染症を伴う.}がある.
\item TBMではBFS所見から\textbf{刀鞘型}と\textbf{三日月型}に分類される.第1〜3度まであり,咳嗽時の内腔狭窄が\textbf{$<$50\%}であれば第1度,\textbf{$>$75\%}であれば第3度.
\item TBMの治療は薬物治療+\textbf{NPPV}.手術する場合は\textbf{膜様部外固定,膜様部縫縮術}.
\item EDAC(過度の動的気道虚脱:excessive dynamic airway collapse)の原疾患は\textbf{喘息}または\textbf{COPD}であり,膜様部の張力が低下し,呼気時に膜様部が過伸展して前方に偏位する.


\end{itemize}

\subsection{血管疾患}
\begin{itemize}
\item 気管支動脈の正常径は\textbf{1.5mm}以下だが,気管支動脈隆を伴った血管は\textbf{2-3mm}に拡張していることが多い.
\item 気管支動脈瘤の原因として,血管疾患に加えて\textbf{気管支拡張症},\textbf{CF},\textbf{珪肺},感染症(\textbf{結核}や\textbf{梅毒})があり得る.
\item 気管支動脈瘤は\textbf{PVへのシャントがなくても良い}が,気管支蔓状血管腫は\textbf{気管支動脈}から「\textbf{PA}または\textbf{PV}」へのシャントの存在が必須である.
\item 気管支蔓状血管腫はその\textbf{R→Lシャント}の特性から,\textbf{奇異性脳塞栓}や\textbf{脳膿瘍}の原因になる.

\end{itemize}

\subsection{気道熱傷と気道外傷}

\begin{itemize}

\item 上気道型気道熱傷は\textbf{喉頭}より上→\textbf{窒息}.下気道型気道熱傷は\textbf{気管}より下→肺実質障害による\textbf{低酸素血症}.
\item 肺実質障害の予測はCXpが有効で,\textbf{肺紋理の増強}→\textbf{すりガラス影}→\textbf{浸潤影}と進むに従って重症.
\item BFSによる気道熱傷の重症度分類は,\textbf{スス沈着,気道分泌,気道狭窄}の程度によってグレード1〜3に分類され,グレード3では\textbf{炎症による組織脆弱性},グレード4では\textbf{粘膜の脱落}や\textbf{内腔の閉塞}を認める.
\item 急性肺障害(ALI)は内視鏡的分類と関連し,\textbf{グレード2}以上でハイリスク.
\item 気道熱傷の喉頭浮腫は\textbf{24時間}がピーク.晩期合併症は気管軟化症,気管狭窄,\textbf{無名動脈気管瘻 (innominate fistula)}がある.気管狭窄は\textbf{気管挿管}の反復回数,期間と相関する.
\item 頸部気道の損傷は鋭的外傷,胸部気道の損傷は鈍的外傷によって起こる.
\item 胸部気道の損傷の原因は,受傷時の\textbf{声門の反射的閉鎖}による急激な気道内圧の上昇(→\textbf{膜様部}が損傷),胸部が前後から潰される(→\textbf{気管分岐部}が損傷する).実際には\textbf{気管分岐部の上下2cm}の損傷が最も多い.
\item 気道外傷は,I度(\textbf{裂傷}),II度(\textbf{不完全断裂}),III度(\textbf{完全断裂})に分類される.
\item 気管損傷が疑われる場合の挿管は,必ず\textbf{BFSガイド下}に行う!重度の気道損傷は開胸手術の適応で,\textbf{単純閉鎖}または\textbf{端々吻合}を行う.


\end{itemize}

\newpage
\section{気管支鏡治療}
\subsection{止血処置}
\begin{itemize}
\item 大出血の場合は出血側(患側)を下にして,健側への片肺挿管を行う.
\item Nd:YAGレーザーは\textbf{易出血性腫瘍の表面}の凝固は可能だが,,\textbf{大血管の破綻}には無力.
\item APCは出血面に対して凝固止血能が高く,Nd:YAGレーザーと比較して深達度が\textbf{浅い}.
\item 高濃度酸素投与下でも使用できるのは\textbf{マイクロ波凝固療法}と\textbf{クライオ}.片肺挿管のときはなんとなく酸素100\% にしてる気がするのでマイクロ波凝固装置は相性が良い気がする…….
\end{itemize}

\subsection{エタノール注入}
\begin{itemize}

\item 中枢気道の狭窄,閉塞には\textbf{99.5\% エタノール(無水エタノール)}を腫瘍内に注入して,腫瘍組織の脱水固定による凝固壊死と,血管の収縮,変性による止血を期待する.ただし即効性はない.
\item エタノールの1回の注入量は\textbf{0.1-0.5mL}で,処置1回あたりの極量は\textbf{3mL}.
\item 注入して数日後にもう一度BFSによる観察を行い,凝固壊死部位を関しで除去する.
\end{itemize}

\subsection{レーザー治療}
\begin{itemize}
\item 適応は\textbf{気管}から\textbf{亜区域気管支の入口部}までの中枢気道であり,マイクロ波凝固療法と異なり水分の少ない組織(\textbf{カフノステノーシス},\textbf{気管支結核による瘢痕})にも適応可.
\item 従来はNd:YAGレーザーが使用されていたが,現在は\textbf{高出力ダイオードレーザー}が一般的である.照射対象から\textbf{5-10mm}離して,\textbf{40-50W}の出力で\textbf{1-2秒}おきの分割照射を行う.
\item マイクロ波凝固療法と同じく,高周波スネアで腫瘍減量を行い,残存腫瘍にレーザーで処理,という手もある.
\end{itemize}

\subsection{高周波治療とAPC}

\begin{itemize}
\item 高周波を発生する器具を病変部に誘導して,焼灼による凝固止血や,組織の切開を行う治療を総称して高周波治療と呼ぶ.
\item 高周波治療の装置には,\textbf{接触型}と\textbf{非接触型}があり,APCは後者,それ以外は前者である.
\item 接触型の高周波治療では,焼灼(凝固止血)と切開のバランスが異なる器具を使い分ける.
\item 高周波ナイフ(いわゆる「電気メス」とほぼ同義):\textbf{切開}がメインで,熱により焼灼も行えるので,切開による出血を減らせる.\textbf{瘢痕性狭窄}によって生じる\textbf{膜様組織}の処置には適するが,壁に垂直に当てると気管支壁穿通のリスクが大きくなる.
\item 高周波スネア:\textbf{切開}と\textbf{焼灼}を同時に担う.ポリープ状の腫瘍は完全にスパッと切ると大出血するので,切開しながら焼灼するのがbest.
\item ホットバイオプシー鉗子:生検のタイミングで通電して\textbf{焼灼}し,出血を抑える.
\item 高周波凝固子:先端が半球状になっており,刃がないので,基本は\textbf{焼灼}のみである.ただし,電圧設定を高くすることで,接触面で爆発的な蒸散を起こし,結果的に腫瘍の体積を減らす(削り取る)ことができる.


\item APCはプローブの先端から出力されるイオン化したアルゴンガスを用いて焼灼を行う.焼灼・凝固された組織抵抗が高くなったところよりも,未治療の組織抵抗が低い方へ流れるので,電流を側面や隅々まで有効に焼灼できるし,気管支壁の穿孔のリスクが小さい.
\item APCに切離,切開のパワーはないが,高周波凝固子と同じ理屈で,局所的な蒸散によって腫瘍減量をはかることも可能である.
\item APCの熱の深達距離は,\textbf{2秒で2mm}.
\item その上で,焼灼による凝固止血の方法として\textbf{ソフト凝固}(低温で深いところまで確実に焼灼・凝固させるので\textbf{深めの血管の処理}に有効),\textbf{フォースド凝固}(一般的な焼灼,イメージは強火で表面を炙る),\textbf{スプレー凝固}(広範囲をシャワーのように焼灼,毛細血管など面状の出血のコーティングのイメージ).

\item まず\textbf{20W}程度で,正常粘膜へのダメージをテストする.
\item 禁忌は\textbf{植え込み式除細動器},\textbf{ペースメーカー}と,酸素濃度を\textbf{40\%}未満に下げられない場合.

\end{itemize}

\subsection{マイクロ波凝固療法}
\begin{itemize}

\item マイクロ波によって\textbf{組織中水分子}を振動させて,腫瘍組織を凝固させる.高濃度酸素投与下でも処置が可能.
\item 組織内水分量が少ない病態(\textbf{カフステノーシス}や\textbf{気管支結核}による瘢痕性狭窄)と,外圧性の気道狭窄には適応外である.
\item プローブへの凝固壊死組織の付着を少なくするために\textbf{解離電流}を用いることもあるが,\textbf{AF}のリスクのため気道には使用できない.そもそも解離電流は\textbf{ペースメーカー}には絶対禁忌.
\item プローブが硬いので\textbf{上葉枝}には使用できない.
\item 凝固条件は,出力\textbf{40-60W}で,\textbf{数秒〜10秒}を1回として繰り返す.むしり取るときに出血のおそれあり.
\item レーザーと同じく,高周波スネアで腫瘍減量を行い,残存腫瘍にマイクロ波凝固療法で処理,という手もある.

\end{itemize}

\section{外科治療}

\begin{itemize}

\item 輪状軟骨の\textbf{腹側}は部分切除可能,背側まで及んでいても,片側であれば同側の\textbf{反回N麻痺}はきたすものの切除・端々吻合が可能である.
\item 気管の端々吻合の技術的限界は\textbf{10軟骨輪(5cm)}.
\item 気管分岐部の切除を行う場合,気管と健側の主気管支を端々吻合するか,\textbf{double-barrel再建}または\textbf{Montage再建}を行う.
\item 気管・気管分岐部形成では,退院までは\textbf{1週間}おきにBFSを行う.吻合線より末梢側の気管支壁の虚血は\textbf{術後7-12日目}に起きやすく,吻合部の血流が完成するのは\textbf{2週間},最終的な治癒に\textbf{7週間}はかかる.
\item 術後の喀痰は\textbf{術後1日目}から増加し,\textbf{術後5日目}頃から減少する.咳ができれば痰が自力排泄できる可能性があるので,場合によっては\textbf{リドカインスプレー}を使わず,\textbf{リドカインゼリー}を鼻腔に塗った状態で経鼻でスコープを挿入し吸痰するという手もある.

\end{itemize}



