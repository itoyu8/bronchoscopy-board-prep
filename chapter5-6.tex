%!TEX root = board-prep.parent.tex

\section{疾患}
\subsection{喉頭〜気管腫瘍}
\begin{itemize}

\item 乳頭腫は\textbf{喉頭}または\textbf{気管〜気管支}に発症する(HPV6/11と関連).気管乳頭腫は\textbf{扁平上皮性},\textbf{腺上皮性},\textbf{混合型}の3つに分かれる.癌化するのは\textbf{扁平上皮性}.

\item 喉頭癌の前癌病変として\textbf{喉頭白板症}があり,CISとの鑑別は\textbf{声帯の可動性}や\textbf{異常血管}の有無に注目する.
\item 喉頭癌の頻度は\textbf{声門},\textbf{声門上},\textbf{声門下}の順.
\item 声帯ポリープは見た目は普通のポリープで,逆にポリープ様声帯は声帯全体が浮腫む.

\item 気管腫瘍で上皮下血管が拡張 ▶ \textbf{平滑筋腫},\textbf{軟骨腫},\textbf{カルチノイド},\textbf{多形腺腫}.
\item 平滑筋腫のマーカーは\textbf{α-SMA},\textbf{カルデスモン},筋原性のマーカーは\textbf{デスミン},\textbf{HHF35}(鑑別としてGISTがあるが,GISTはこれらは陰性).
\item 気管腫瘍で有茎状 ▶ \textbf{脂肪腫}.
\item 気管〜気管支腫瘍でS-100蛋白陽性 ▶ \textbf{顆粒細胞腫}または\textbf{神経原性腫瘍}.
\item 顆粒細胞腫はその名の通りBFS所見は\textbf{凹凸不整(顆粒状)}.悪性転化したり重複癌のメルクマールになる.
\item 気管腫瘍で「硬く生検困難」▶ \textbf{軟骨腫},\textbf{脂肪腫}(表面の被膜が硬いので).

\item 腺様嚢胞癌は組織学的には\textbf{唾液腺型腫瘍}であり,\textbf{管腔型}より\textbf{充実型}で予後不良.

\item 気管支腫瘍は\textbf{良性と分かっていれば内視鏡的治療(YAG,スネア)}.逆に外科的治療の条件は,①悪性の可能性あり,②腫瘍による気道閉塞が高度,③気管支壁への腫瘍浸潤.例外的に\textbf{多形腺腫}はBFSで治療が完結できる!
\end{itemize}
\subsection{気管支〜肺の悪性腫瘍}

\begin{itemize}

\item ポリープ型の悪性腫瘍の鑑別として\textbf{早期Sq},\textbf{大細胞癌},\textbf{多形腺腫},\textbf{平滑筋肉腫}.

\item ★内視鏡的早期肺癌の基準:\textbf{気管から亜区域枝}\footnote{つまりB1aまで.}までに限局,\textbf{腫瘍の末梢辺縁が内視鏡的に可視},\textbf{長径$<$2cm},\textbf{組織学的にSq}.
\item 内視鏡的早期肺癌は\textbf{平坦型},\textbf{結節型},\textbf{早期ポリープ型}に分類される.平坦型においてCISの可能性が高いのは\textbf{浸潤範囲$<$10mm}.
\item 進行Sqは\textbf{粘膜型},\textbf{粘膜下型},\textbf{壁外型}に分類される.
\item SCLCは\textbf{上皮下型}または\textbf{壁内型}で多彩な病像を呈するが,前者であれば上皮下血管の消失,後者であれば縦走襞の圧縮強調を伴う.


\item 甲状腺癌の気管浸潤は\textbf{気管壁に接して易出血性の発赤}を認める.食道癌の気管浸潤はどちらかというと\textbf{亜全周性}の病変になることが多い.

\item 肺転移の鑑別で,CK7+CK20+なら\textbf{尿路上皮癌,膵癌},CK7-CK20+なら\textbf{大腸癌},CK7-CK20-なら\textbf{肝細胞癌,前立腺癌,腎細胞癌}.
\item メラノーマの気管転移は\textbf{気管支内腔のポリープ状の発育}で,表面は\textbf{青みを帯びた黒色調}のことが多い(ただし転移病変がメラニンを含まないことはよくある).

\item 肺癌以外の気管支悪性腫瘍で,完全切除が可能なのは\textbf{平滑筋肉腫}と\textbf{多形腺腫}.
\item 出血しやすい悪性腫瘍 ▶ \textbf{腎癌の気管支転移}.
\end{itemize}




\subsection{結核}
\begin{itemize}

\item 喉頭結核で最も多いのは\textbf{声帯$>$喉頭蓋$>$仮声帯}で,肉芽腫型が多い.
\item 気管・気管支結核の定義は,「\textbf{区域気管支}\footnote{つまりB1まで.}より中枢側の気管・気管支の結核性粘膜病変」で,\textbf{左主気管支}に好発する.
\item 気管支結核の内視鏡分類(荒井分類):早期病変は\textbf{Type II(粘膜内結節型)}である.白苔を伴う潰瘍形成がある場合は\textbf{Type III(潰瘍型)}で,隆起性潰瘍であれば\textbf{Type IIIb}.
\item 結核の治癒過程に入ると\textbf{Type IV(肉芽型)}になり,最終的に\textbf{Type V(瘢痕型)}になる.狭窄を伴う場合は\textbf{Type Vb}.

\item 結核性リンパ節炎が気管支内に穿破すると\textbf{Type LN}で,古典的には小児・若年成人に多い.

\end{itemize}


\subsection{真菌感染症}

\begin{itemize}

\item ABPMでの病理では\textbf{Charcot-Leyden結晶}(好酸球性炎症を反映するタンパク質が結晶化)が特徴的.
\item PCPのBALFでは\textbf{卵円形嚢子}がクラスターで観察されれば確定.
\end{itemize}

\subsection{アレルギー・間質性肺疾患}
\begin{itemize}

\item 喘息患者のうち,好中球性の気道炎症が強い場合は\textbf{Th17}が関与する.ステロイド抵抗性の重症喘息では\textbf{Th2/Th17 positive}の細胞増生がある.
\item 喘息患者のBALFではLT,PG,TX,ヒスタミン,\textbf{トリプターゼ}などのメディエーターに加えて,\textbf{TGF-β1}や\textbf{bFGF}の増加がみられる.
\item HPでは夏型を除いてCD4/CD8比は\textbf{上昇}.CHPでも\textbf{上昇}.
\item 他にCD4/CD8比の上昇は\textbf{MTX肺炎},\textbf{サルコイドーシス},\textbf{慢性ベリリウム肺}.

\item 血管炎のうちGPAはBFSの可視範囲において発赤,腫脹を広範囲に認める.\textbf{声門下狭窄}も17\% に認める.
\item 血管炎のBALFは,MPAは特徴的な所見はなく,GPAは\textbf{Ny⧺,Eo+},EGPAでは\textbf{Eo⧺}.
\item 名前の通り,GPAの病理では\textbf{肉芽腫→壊死},\textbf{血管炎}がすべて揃う.MPAの病理は基本的に肉芽腫はない.

\end{itemize}

\subsection{サルコイドーシス}

\begin{itemize}

\item サルコイドーシスのBFS所見:\textbf{上皮下血管の増生}(いわゆるnetwork formation)と,上皮組織の混濁による\textbf{上皮下血管の消失}のどっちもあり!内腔の\textbf{顆粒状変化}は若年層に多い.
\item サルコイドーシスのI期は\textbf{BHL},II期は\textbf{BHL+肺陰影},III期は\textbf{肺陰影 alone},IV期は\textbf{肺線維化}(LN→肺実質と炎症が移行する).\end{itemize}

\subsection{アミロイドーシス}

\begin{itemize}

\item アミロイドーシスのBFS所見では,\textbf{プラーク}または\textbf{隆起性病変}が飛び飛びに見られることに加えて,\textbf{血管増生}が目立ち,大出血することもある!
\item 気管・気管支アミロイドーシスでは\textbf{AL}が多く,全身性アミロイドーシスの気道病変として現れる場合は\textbf{AA}が多い.進行例では気管支狭窄が起こり得る.
\item 肺野結節型のアミロイドーシスはAA, ALのいずれでも起こる.
\item びまん性AL型アミロイドーシスの肺病変ととして\textbf{びまん性肺胞隔壁アミロイドーシス}があり,小葉間隔壁の肥厚などIP likeな画像所見を呈する.
\end{itemize}

\subsection{稀な疾患}

\begin{itemize}
\item LCHは進行すると組織中のLangerhans細胞は消失する.細気管支周辺に\textbf{stellate fibrotic scar}を認める.BALFではCD4/CD8は\textbf{低下}.



\item 石綿肺のBALFにおいて肺癌発症リスクを高める所見として\textbf{BALF 1mLあたり石綿小体が5本以上}がある.
\item 珪肺のBALFでは\textbf{Ly+}を認め,CD4/CD8比は\textbf{低下}する.
\item 溶接工肺は,BALFまたはTBLBで\textbf{鉄染色陽性のMph}を認める.

\item PAPのBALFでは,\textbf{顆粒状の無構造物質}の沈着と,\textbf{泡沫状Mph}がみられ,これらはPapanicolaou染色で\textbf{ライトグリーン}に染まる.
\item PAPの病理所見では,末梢気腔内に\textbf{弱好酸球性細顆粒状物質}を認め,これは\textbf{PAS}染色で陽性,\textbf{SP-A}にも陽性である.随伴所見として,末梢気腔内に\textbf{大型泡沫細胞}が集積する.

\item LAMのBALFでは\textbf{ヘモジデリン貪食像を伴うMph}の増加を認める.LAM細胞は平滑筋の性質を持つので\textbf{HMB-45}が陽性であれば確定診断(ただしLAM細胞は集塊で認めるので,すべてのTBLB検体で見られるわけではない).LAM細胞は\textbf{PgR}が陽性のものもある.
\item LAMでは血清で\textbf{VEGF-D}の上昇が診断に有用(保険未収載).

\item 気管支結石は,古典的には\textbf{結核}や\textbf{ヒストプラズマ}との関連が重要.\textbf{管内性}(気管支内部で発育)または\textbf{管外性}(気管支周囲リンパ節の気管支内穿孔).
\end{itemize}



\subsection{気管・気管支骨軟骨形成症(TO)}

\begin{itemize}

\item 気管・気管支骨軟骨形成症(TO)は,気道の\textbf{粘膜下組織}に,骨・軟骨組織が異所性に増生する.
\item 好発部位は\textbf{気管近位部〜中間部}で,\textbf{主気管支}くらいまで病変がある.ただし,\textbf{膜様部}には病変はない.
\item TOの組織所見は\textbf{扁平上皮化生}と,骨・軟骨,石灰化,\textbf{骨化領域の造血骨髄}をみる.
\item CTで多発石灰化病変を認めた場合,鑑別診断は\textbf{再発性多発軟骨炎},\textbf{サルコイドーシス},\textbf{アミロイドーシス},\textbf{乳頭腫}などが主な鑑別となる.


\end{itemize}

\subsection{気管・気管支軟化症}

\begin{itemize}

\item 気管・気管支軟化症(TBM)は高度の咳発作の反復によっても起こるが,secondaryに起こるものとして\textbf{軟骨損傷},\textbf{気管・気管支結核},\textbf{再発性多発軟骨炎},\textbf{Mounier-Kuhn症候群}\footnote{気管気管支の内腔が異常に拡張する稀な疾患で,しばしば慢性呼吸器感染症を伴う.}がある.
\item TBMではBFS所見から\textbf{刀鞘型}と\textbf{三日月型}に分類される.第1〜3度まであり,咳嗽時の内腔狭窄が\textbf{$<$50\%}であれば第1度,\textbf{$>$75\%}であれば第3度.
\item TBMの治療は薬物治療+\textbf{NPPV}.手術する場合は\textbf{膜様部外固定,膜様部縫縮術}.
\item EDAC(過度の動的気道虚脱:excessive dynamic airway collapse)の原疾患は\textbf{喘息}または\textbf{COPD}であり,膜様部の張力が低下し,呼気時に膜様部が過伸展して前方に偏位する.


\end{itemize}

\subsection{血管疾患}
\begin{itemize}
\item 気管支動脈の正常径は\textbf{1.5mm}以下だが,気管支動脈隆を伴った血管は\textbf{2-3mm}に拡張していることが多い.
\item 気管支動脈瘤の原因として,血管疾患に加えて\textbf{気管支拡張症},\textbf{CF},\textbf{珪肺},感染症(\textbf{結核}や\textbf{梅毒})があり得る.
\item 気管支動脈瘤は\textbf{PVへのシャントがなくても良い}が,気管支蔓状血管腫は\textbf{気管支動脈}から「\textbf{PA}または\textbf{PV}」へのシャントの存在が必須である.
\item 気管支蔓状血管腫はその\textbf{R→Lシャント}の特性から,\textbf{奇異性脳塞栓}や\textbf{脳膿瘍}の原因になる.

\end{itemize}

\subsection{気道熱傷と気道外傷}

\begin{itemize}

\item 上気道型気道熱傷は\textbf{喉頭}より上→\textbf{窒息}.下気道型気道熱傷は\textbf{気管}より下→肺実質障害による\textbf{低酸素血症}.
\item 肺実質障害の予測はCXpが有効で,\textbf{肺紋理の増強}→\textbf{すりガラス影}→\textbf{浸潤影}と進むに従って重症.
\item BFSによる気道熱傷の重症度分類は,\textbf{スス沈着,気道分泌,気道狭窄}の程度によってグレード1〜3に分類され,グレード3では\textbf{炎症による組織脆弱性},グレード4では\textbf{粘膜の脱落}や\textbf{内腔の閉塞}を認める.
\item 急性肺障害(ALI)は内視鏡的分類と関連し,\textbf{グレード2}以上でハイリスク.
\item 気道熱傷の喉頭浮腫は\textbf{24時間}がピーク.晩期合併症は気管軟化症,気管狭窄,\textbf{無名動脈気管瘻 (innominate fistula)}がある.気管狭窄は\textbf{気管挿管}の反復回数,期間と相関する.
\item 頸部気道の損傷は鋭的外傷,胸部気道の損傷は鈍的外傷によって起こる.
\item 胸部気道の損傷の原因は,受傷時の\textbf{声門の反射的閉鎖}による急激な気道内圧の上昇(→\textbf{膜様部}が損傷),胸部が前後から潰される(→\textbf{気管分岐部}が損傷する).実際には\textbf{気管分岐部の上下2cm}の損傷が最も多い.
\item 気道外傷は,I度(\textbf{裂傷}),II度(\textbf{不完全断裂}),III度(\textbf{完全断裂})に分類される.
\item 気管損傷が疑われる場合の挿管は,必ず\textbf{BFSガイド下}に行う!重度の気道損傷は開胸手術の適応で,\textbf{単純閉鎖}または\textbf{端々吻合}を行う.


\end{itemize}

\newpage


\section{気管支鏡治療}
\subsection{止血処置}
\begin{itemize}
\item 大出血の場合は出血側(患側)を下にして,健側への片肺挿管を行う.
\item Nd:YAGレーザーは\textbf{易出血性腫瘍の表面}の凝固は可能だが,,\textbf{大血管の破綻}には無力.
\item APCは出血面に対して凝固止血能が高く,Nd:YAGレーザーと比較して深達度が\textbf{浅い}.
\item 高濃度酸素投与下でも使用できるのは\textbf{マイクロ波凝固療法},\textbf{クライオ},\textbf{PDT}.片肺挿管のときはなんとなく酸素100\% にしてる気がするのでマイクロ波凝固装置は相性が良い気がする…….
\end{itemize}

\subsection{エタノール注入}
\begin{itemize}

\item 中枢気道の狭窄,閉塞には\textbf{99.5\% エタノール(無水エタノール)}を腫瘍内に注入して,腫瘍組織の脱水固定による凝固壊死と,血管の収縮,変性による止血を期待する.ただし即効性はない.
\item エタノールの1回の注入量は\textbf{0.1-0.5mL}で,処置1回あたりの極量は\textbf{3mL}.
\item 注入して数日後にもう一度BFSによる観察を行い,凝固壊死部位を関しで除去する.
\end{itemize}

\subsection{レーザー治療}
\begin{itemize}
\item 適応は\textbf{気管}から\textbf{亜区域気管支の入口部}までの中枢気道であり,マイクロ波凝固療法と異なり水分の少ない組織(\textbf{カフノステノーシス},\textbf{気管支結核による瘢痕})にも適応可.
\item 機械的切除の併用により効率が上がるため,\textbf{硬性鏡}下の施行も選択肢である.
\item 従来はNd:YAGレーザーが使用されていたが,現在は\textbf{高出力ダイオードレーザー}が一般的である.照射対象から\textbf{5-10mm}離して,\textbf{40-50W}の出力で\textbf{1-2秒}おきの分割照射を行う.
\item 全集型や隆起型に関しては中心部,または狭窄の強い部位から焼灼し,有茎型は基部の焼灼(およびそれによる切断)を狙う.
\item マイクロ波凝固療法と同じく,高周波スネアで腫瘍減量を行い,残存腫瘍にレーザーで処理,という手もある.
\item 処置時の合併症として,ピンポイント照射による\textbf{気管支穿孔}および\textbf{血管穿孔}(致命的!!),組織焼灼時の煤煙による\textbf{呼吸不全}がある.スネアやcore outを併用し,可能な限り照射時間を短縮する.
\item 長期的な合併症として,レーザー照射によって熱変性による\textbf{瘢痕狭窄}を起こす.
\end{itemize}

\subsection{PDT}
\begin{itemize}

\item \textbf{ポルフィリン関連化合物}(レザフィリン$^{\circledR}$,664nmに吸収スペクトルがある)と低出力レーザー(PDレーザ$^{\circledR}$,664nm,\textbf{赤色})を組み合わせて,選択的な殺細胞効果をもたらす.中心型早期肺癌に対する根治的治療,進行肺癌に対する姑息的治療の両方に適応がある.

\item ★中心型早期肺癌の定義:\textbf{区域気管支}より中枢に位置し,癌の浸潤が組織学的に\textbf{気管支壁}を超えず,リンパ節転移および遠隔転移がない.\footnote{ (再掲)内視鏡的早期肺癌:{気管から亜区域枝}までに限局,{腫瘍の末梢辺縁が内視鏡で可視},{長径$<$2cm},{組織学的にSq}.see \S 4.2.
}
\item ★PDTの適応がある中心型早期肺癌:\textbf{腫瘍長径$<$10mm}\footnote{腫瘍長径$<$10mmで,内視鏡的に平坦型であればCISの可能性が高い.see \S 4.2.},深達度が\textbf{上皮下層}まで.ただし,腫瘍長径$>$10mmであっても,内視鏡的に末梢側の辺縁が確認できればPDT施行可.
\item レザフィリン投与後,\textbf{約10秒後}から\textbf{SpO2低下}(SpO2モニターの赤色光とレザフィリンが干渉するため).

\item レザフィリン投与後\textbf{4-6時間後}にレーザー照射を施行する.
\item PDTの照射条件は,\textbf{100J/cm2},\textbf{120mW}で\textbf{11分7秒}.
\item PDT中はスコープ操作に注意し\textbf{出血}を防ぐ(Hbに赤色光が吸収されてPDT効率が落ちる).
\item PDT終了後,\textbf{AFB}で観察すると,もし癌が残っている場合は,癌病巣内のレザフィリンが励起されて,\textbf{赤色光}を発生する(\textbf{Photo-dynamic diagnosis: PDD}).
\item PDTは\textbf{気管支穿孔}のリスクがきわめて小さく,RTと異なり\textbf{繰り返しの治療}が可能で,\textbf{HOT}使用中の患者にも適応がある.
\item レザフィリン使用後,1週間程度は\textbf{直射日光}を避ける.
\item 進行肺癌に対するPDTは,Nd:YAGレーザーと比べて腫瘍退縮までの時間は\textbf{長い}が,再狭窄までの期間はNd:YAGレーザーと比べて\textbf{数ヶ月長い}.


\end{itemize}

\subsection{高周波治療とAPC}

\begin{itemize}
\item 高周波を発生する器具を病変部に誘導して,焼灼による凝固止血や,組織の切開を行う治療を総称して高周波治療と呼ぶ.
\item 高周波治療の装置には,\textbf{接触型}と\textbf{非接触型}があり,APCは後者,それ以外は前者である.
\item 接触型の高周波治療では,焼灼(凝固止血)と切開のバランスが異なる器具を使い分ける.
\item 高周波ナイフ(いわゆる「電気メス」とほぼ同義):\textbf{切開}がメインで,熱により焼灼も行えるので,切開による出血を減らせる.\textbf{瘢痕性狭窄}によって生じる\textbf{膜様組織}の処置には適するが,壁に垂直に当てると気管支壁穿通のリスクが大きくなる.
\item 高周波スネア:\textbf{切開}と\textbf{焼灼}を同時に担う.ポリープ状の腫瘍は完全にスパッと切ると大出血するので,切開しながら焼灼するのがbest.
\item ホットバイオプシー鉗子:生検のタイミングで通電して\textbf{焼灼}し,出血を抑える.
\item 高周波凝固子:先端が半球状になっており,刃がないので,基本は\textbf{焼灼}のみである.ただし,電圧設定を高くすることで,接触面で爆発的な蒸散を起こし,結果的に腫瘍の体積を減らす(削り取る)ことができる.


\item APCはプローブの先端から出力されるイオン化したアルゴンガスを用いて焼灼を行う.焼灼・凝固された組織抵抗が高くなったところよりも,未治療の組織抵抗が低い方へ流れるので,電流を側面や隅々まで有効に焼灼できるし,気管支壁の穿孔のリスクが小さい.
\item APCに切離,切開のパワーはないが,高周波凝固子と同じ理屈で,局所的な蒸散によって腫瘍減量をはかることも可能である.
\item APCの熱の深達距離は,\textbf{2秒で2mm}.
\item その上で,焼灼による凝固止血の方法として\textbf{ソフト凝固}(低温で深いところまで確実に焼灼・凝固させるので\textbf{深めの血管の処理}に有効),\textbf{フォースド凝固}(一般的な焼灼,イメージは強火で表面を炙る),\textbf{スプレー凝固}(広範囲をシャワーのように焼灼,毛細血管など面状の出血のコーティングのイメージ).

\item まず\textbf{20W}程度で,正常粘膜へのダメージをテストする.
\item 禁忌は\textbf{植え込み式除細動器},\textbf{ペースメーカー}と,酸素濃度を\textbf{40\%}未満に下げられない場合.

\end{itemize}

\subsection{マイクロ波凝固療法}
\begin{itemize}

\item マイクロ波によって\textbf{組織中水分子}を振動させて,腫瘍組織を凝固させる.高濃度酸素投与下でも処置が可能.
\item 組織内水分量が少ない病態(\textbf{カフステノーシス}や\textbf{気管支結核}による瘢痕性狭窄)と,外圧性の気道狭窄には適応外である.
\item プローブへの凝固壊死組織の付着を少なくするために\textbf{解離電流}を用いることもあるが,\textbf{AF}のリスクのため気道には使用できない.そもそも解離電流は\textbf{ペースメーカー}には絶対禁忌.
\item プローブが硬いので\textbf{上葉枝}には使用できない.
\item 凝固条件は,出力\textbf{40-60W}で,\textbf{数秒〜10秒}を1回として繰り返す.むしり取るときに出血のおそれあり.
\item レーザーと同じく,高周波スネアで腫瘍減量を行い,残存腫瘍にマイクロ波凝固療法で処理,という手もある.

\end{itemize}

\subsection{放射線治療:腔内照射}

\begin{itemize}
\item 気管支腔内照射は,\textbf{内視鏡的早期肺癌}に対して適応がある.
\item 緩和照射を除き,1回線量の上限は\textbf{6Gy}.
\item 早期肺癌に対しては,外照射\textbf{40Gy}+腔内照射\textbf{6Gy}*\textbf{3回}(週1回) = 58Gy.
\item それ以外の肺癌に対しては,外照射\textbf{60Gy}+腔内照射\textbf{6Gy} * \textbf{2-3回}(週1回).
\item 緩和照射では,\textbf{10Gy}の単回照射で,基本的に外照射の併用は\textbf{行わない}.
\item イントラルミナルカテーテル(ILC)の外側に\textbf{アプリケータ}を通し,アプリケータの先端近くの2つのウイングが開いた状態を確認して,腔内照射を行う.
\end{itemize}

\subsection{ゴールドマーカー留置術}

\begin{itemize}
\item ゴールドマーカーは\textbf{4個}留置し,胸壁から\textbf{1cm以内},病変の中心から\textbf{5cm以内}かつ辺縁から\textbf{3cm以内},マーカー同士は\textbf{2cm以上}離す.
\item ゴールドマーカーが脱落しやすい部位は\textbf{左上葉}.

\end{itemize}

\subsection{気道ステント留置術}
\begin{itemize}

\item 良性気道狭窄には\textbf{シリコンステント}を用いる(長期留置は望ましくなく,金属ステントは抜去不可能のため).外科的治療ができればそれがbest.
\item 食道気管支瘻や気管縦隔瘻には\textbf{covered typeの金属ステント}を用いる(密閉性が高い).
\item 硬性鏡の使用が必須なのは\textbf{シリコンステント}で,金属ステントやハイブリッドステントは軟性鏡でも挿入可.
\item 声門直下の狭窄には\textbf{Tチューブ}を用いるが,外科的な気管孔作成が必要.
\item Yステントは気管分岐部のみならず,右上幹-中間幹(\#11s),左上幹-下幹(\#11L)の病変にも対応できる.
\item 金属ステントはcovered typeであっても,両端に長さ\textbf{7.5mm}の膜のない部分があり,金属部分が上皮や肉芽に覆われるので,抜去は不可.

\end{itemize}

\subsection{気管支サーモプラスティ(BT)}
\begin{itemize}
\item BTの術前準備として,術前3日〜翌日までPSL50mg/dを内服し,\textbf{手技当日,気管支拡張薬吸入後のFEV$_{1.0}$が,平時の85\%以上}を確認する.
\item \textbf{右下葉},\textbf{左下葉},\textbf{両側上葉}の3回に分けて,\textbf{3週間}以上の間隔を空けて行う.
\item \textbf{右中葉}は気管支が長く中葉症候群の発症リスクのため,BTは行わない.
\item 末梢側から\textbf{10秒}ずつ通電を与えて,\textbf{5mm}ずつ引きながら通電を繰り返す.内径が\textbf{10mm}を超える場合,プローブが気管支面に接触できないので,処置は不可能.
\item 治療後は,処置気管支の壁肥厚,気管支周囲のconsolidation,無気肺がみられるが,3週間でおおむね消失する.
\end{itemize}

\subsection{COPDの内視鏡的治療}
\begin{itemize}

\item COPDに対する気管支鏡的肺容量減量術(BLVR)は,\textbf{一方向弁}または\textbf{形状記憶コイル}を用いる.
\item BLVRの適応条件は,\textbf{禁煙}が絶対条件で,\%FEV1が\textbf{20-45\%},\%RV $>$175\%,\%RV/TLC $>$58\%,6分間歩行距離が\textbf{100-500m}.
\item 主な除外基準として,気管支拡張症,\textbf{胸部手術後},PaCO2$>$60Torr,PaO2$<$45Torr,\%DLco$<20$\%.
\item BLVRが施行できるためには,側副換気がないことが前提であり,CTで葉間胸膜が\textbf{95\%}であればOK,80-95\%はborderline.
\end{itemize}

\subsection{小児の気管支鏡}
\begin{itemize}
\item 処置孔を持った最小径の軟性鏡は,外径2.8mm,鉗子口1.2mmで,硬性鏡は外径7mm,内径6.5mm.これらを組み合わせることで,異物鉗子も入れることができる.
\item ラリンジアルマスクの使用で気道径を確保する(テキスト内でもばらついているが,\textbf{2歳}以上または\textbf{5kg}以上なら使用可能).
\item 小児の肺外気管支は柔らかく,特に\textbf{左主気管支}の閉塞には要注意.
\end{itemize}

\subsection{その他の処置}
\begin{itemize}
\item EWSは難治性気胸に対する処置で,責任気管支を葉支→区域支→亜区域支の順に15-20秒バルーンで閉塞させ,ドレーンからの気瘻が現症(消失)すればその部分を詰める.
\item 気管支瘻の閉鎖には,まずフィブリン糊などで病変部を塞ぎ,その上から\textbf{PGAシート}や\textbf{シリコン}を詰める.瘻孔の大きさが\textbf{8mm}を超える場合,一般的にはBFSによる閉鎖は不可能.
\item 異物除去は\textbf{バルーンカテーテル}を末梢に誘導して中枢気管支に引きずり出すのが簡便.
\item 挿管によって処置に使える内腔が狭小化するため,特に小児,高齢者では\textbf{ラリンジアルマスク}を利用した方が良い.


\end{itemize}


\subsection{外科的治療}

\begin{itemize}

\item 輪状軟骨の\textbf{腹側}は部分切除可能,背側まで及んでいても,片側であれば同側の\textbf{反回N麻痺}はきたすものの切除・端々吻合が可能である.
\item 気管の端々吻合の技術的限界は\textbf{10軟骨輪(5cm)}.
\item 気管分岐部の切除を行う場合,気管と健側の主気管支を端々吻合するか,\textbf{double-barrel再建}または\textbf{Montage再建}を行う.
\item 気管・気管分岐部形成では,退院までは\textbf{1週間}おきにBFSを行う.吻合線より末梢側の気管支壁の虚血は\textbf{術後7-12日目}に起きやすく,吻合部の血流が完成するのは\textbf{2週間},最終的な治癒に\textbf{7週間}はかかる.
\item 術後の喀痰は\textbf{術後1日目}から増加し,\textbf{術後5日目}頃から減少する.咳ができれば痰が自力排泄できる可能性があるので,場合によっては\textbf{リドカインスプレー}を使わず,\textbf{リドカインゼリー}を鼻腔に塗った状態で経鼻でスコープを挿入し吸痰するという手もある.

\end{itemize}



