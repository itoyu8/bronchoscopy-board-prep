%!TEX root = board-prep.parent.tex



\section{基本事項}
\begin{itemize}
\item 気管支ビデオスコープの先端には\textbf{CCD (charged coupled device)}が装着される.CCDの小型化には限界があるため,先端に光ファイバーを装着し,スコープ内にCCDを備え付ける\textbf{ハイブリッドスコープ}も

\end{itemize}

\section{気管支鏡の観察項目}
\subsection{所見}
\begin{itemize}

\item 内視鏡的層別分類は,内側から\textbf{上皮層}(上皮+基底膜),\textbf{上皮下層}(上皮化血管,縦走襞),\textbf{壁内層}(平滑筋,軟骨),\textbf{壁外層}の4つである.

\item 縦走襞は\textbf{上皮下層の深部の弾力線維}であり,輪状襞は\textbf{壁内層の平滑筋}である.

\item 気管支表面の白色調の変化は\textbf{萎縮性気管支炎,腫瘍の上皮直下への進展,貧血}で認める.
\item 気管支内腔における黒褐色の色調変化は\textbf{上皮下}の炭粉沈着を反映する.
\item 気道狭窄のうち,外からの圧排+内腔内にも所見を認めるものは\textbf{混合性狭窄}である.
\item 気道狭窄のうち,ステント治療の適応から外れるのは\textbf{腔内性狭窄}である.
\item 機能性狭窄は\textbf{気管・気管支軟化症},\textbf{excessive dynamic airway collapse (EDAC)}などの支持組織の異常による.

\item 隆起性変化のうち,$>$2mmの高さを\textbf{結節}と呼ぶ.$<$2mmの変化は\textbf{平坦性変化}と呼ぶ.
\item 平坦性変化のうち,「肥厚」は\textbf{気管支分岐部}における所見として用いる.\textbf{Sq}が代表的である.
\item 平坦性変化のうち,気管支表面の凹凸不整な変化を\textbf{顆粒状(凹凸不整)}とよぶ.肺癌などの悪性疾患のほか,\textbf{サルコイドーシス}でも認める.
\item 平坦性変化のうち,上皮組織の欠損による\textbf{陥凹性変化}は,\textbf{気管支結核},\textbf{Sq}の一部で認める.

\item 上皮下血管の増生は,\textbf{肺癌のリンパ節転移},\textbf{縦隔腫瘍},\textbf{サルコイドーシス},\textbf{SVC症候群}などでみられる.
\item 軟骨・軟骨輪の増生性変化は,\textbf{気管気管支骨軟骨形成症}で認める.

\end{itemize}
\subsection{腫瘍の深達度診断}
\begin{itemize}


\item \textbf{上皮下血管の消失}がある場合,腫瘍は\textbf{上皮層}に存在する.
\item 腫瘍の\textbf{表面の色調変化}がある場合,少なくとも腫瘍は\textbf{上皮下層}以浅にある.特に,上皮下浅層まで及べば縦走襞が\textbf{消失}する.
\item 腫瘍が壁内層に広く浸潤する場合,\textbf{輪状襞}は消失し,場合によって\textbf{縦走襞の圧縮強調}がみられる.\textbf{Ad, SCLC}に典型的であり,切除範囲は慎重に決定する!

\item bridging foldは,\textbf{壁外層}または\textbf{壁内層}から圧排を受けることで,縦走襞が盛り上がって見える所見である.bridging foldの途中で縦走襞が断裂していれば,\textbf{上皮下浅層}までの進展を意味する.

\item 壁内層または壁外層の腫瘍が気管・気管支壁に浸潤しているか判断したい場合,\textbf{EBUS}を使う.上皮を主体とした病変で境界がわかりにくい場合,\textbf{AFI}や\textbf{NBI}を用いる.


\end{itemize}

\section{処置と検体処理}
\subsection{R-EBUS}

\begin{itemize}
\item R-EBUSを用いて\textbf{中枢側}の気管支壁の深達度診断を行う場合,プローブをバルーンシースで膨らませて空気を抜き,超音波画像を得る.
\item 膜様部は3層構造,そのほかの肺外気管支と肺内気管支は5層構造.
\item 肺外気管支の5層構造は,内側から境界エコー(high),上皮下組織(low),気管支軟骨の内側縁(high),気管支軟骨(low),気管支軟骨の外側縁(high).気管支軟骨より浅い病変でPDTが有効なので,\textbf{第4層(気管支軟骨のローエコー)}を拾うことが大切.
\item EBUS-GSによる生検回数は\textbf{5-6回}でピークに近づく.
\end{itemize}

\subsection{EBUS-TBNA}
\begin{itemize}
\item TBNAスコープを食道に挿入することも可能で,EUS-B-FNAと呼ぶ.\textbf{\#8, \#9}はEUS-B-FNAの絶対適応で.\textbf{\#5, \#7}の一部もEUS-B-FNAでしか生検できないことがある.
\item 腫瘍のリンパ節ステージングでは,\textbf{3 station}以上の生検が必須で,その中に\textbf{\#4R, \#4L, \#7}を含める.
\item リンパ節の左右の血管は,\#2Rが\textbf{SVC-BCA},\#4Rが\textbf{Az-SVC},\#4Lが\textbf{PA-Ao}.
\item EBUS-TBNAの穿刺において\textbf{陰圧は必須ではない.}ROSEをしない場合,同一部位に対して最低\textbf{3回}は穿刺する.
\item 転移のないリンパ節は,形状は楕円形,辺縁は\textbf{不明瞭}で,内部は中心部の\textbf{高エコー}を伴い,内部血流は\textbf{少なく},\textbf{リンパ節門の血流}が観察される.逆に,転移性のリンパ節は\textbf{気管支動脈}からの流入血管を認める.
\end{itemize}

\subsection{AFB(Autofluorescence bronchoscopy:自家蛍光気管支鏡)}

\begin{itemize}

\item AFBは普通のBFSでは気管支壁に紛れて見逃してしまうような,\textbf{早期癌病変}を見つけるモダリティである.

\item \textbf{青色}(420-460nm)の励起光を気管支の正常部に照射すると,\textbf{緑色}(480-520nm)の自家蛍光が発生するが,腫瘍部ではこの波長の自家蛍光が減弱するので,正常部とのコントラストを増幅して観察できる.
\item AFBの偽陽性となるのは\textbf{慢性炎症}や\textbf{血管の増生}である.
\end{itemize}

\subsection{NBI(Narrow band imaging:狭帯域光観察)}
\begin{itemize}
\item NBIは\textbf{毛細血管を強調}することで,特に\textbf{Sq}のような多段階発癌の様子を推定できる.
\item 気管支の正常部は\textbf{青色}と\textbf{緑色}の光を反射するが,逆にHbはこれらの色を吸収する(特に青色を強く吸収する).
\item \text{上皮〜上皮下表層}の血管像が\textbf{茶色}で,表層下のやや深い血管が\textbf{緑色〜シアン色}に見える.
\item 気管支Sqは,\textbf{squamous dysplasia},\textbf{ASD (angiogenic squamous dysplasia)},\textbf{CIS (carcinoma in situ,上皮内癌)},微小浸潤癌,浸潤癌と進展していく.
\item NBIにおいて,微細な血管網の増生,蛇行がみられたら,\textbf{squamous dysplasia}または\textbf{ASD}を考える.点状血管があれば\textbf{ASD}以上で,いよいよらせん型・スクリュー型の血管がみられると\textbf{CIS}以上と考える.



\end{itemize}

\subsection{その他のモダリティ}

\begin{itemize}

\item 超細径気管支鏡は外径\textbf{3.5mm}以下で,処置チャンネルは\textbf{1.7mm}が主流となっている.基本は経鼻挿管で,\textbf{カフなしφ5mm}のチューブを先に挿入しておく.

\item \textbf{EMN (electromagnetic navigation)}は,位置センサーを内蔵したガイドシースをスコープに通した状態で,リアルタイムなスコープの先端位置を再構成画像上に描出する.位置センサーが目標位置に到達したらセンサーを抜いてガイドシースを残し,あとは普通の生検と同じ.
\item \textbf{CLE (confocal laser endomicroscopy)}は,末梢型肺腫瘍において,ROSEと同じような細胞観察を,BFS越しに行う.細径シースにミニプローブ(φ1.4mm)を挿入して腫瘍まで誘導したのち,\textbf{フルオレセイン}を注射して観察する.
\item OCTは\textbf{上皮層}と\textbf{上皮下層}までは観察できるので,末梢のCISの診断などに応用可能.
\end{itemize}

\subsection{検体採取と処理}
\begin{itemize}

\item TBB/TBLBは検体採取終了後,\textbf{2-3分}はwedgeし,検査\textbf{2時間後}にXp撮影.
\item キュレットは中枢気道病変でスコープのアングルを目一杯使っても到達できない部位の擦過も可能である.
\item 気管支洗浄を行うときは\textbf{リドカイン}はなるべく使用しない(殺菌作用があるため).
\item クライオプローブの外径はφ2.4mmとφ1.9mmが一般的.末梢肺病変の生検を行う場合,φ1.1mmのプローブで気管支鏡を抜かずに生検する(miniCryo)も有望.
\item 細胞診検体(特にTBNA検体)の場合,血液成分が多いので,\textbf{スライドガラスを擦り合わせて}検体を伸ばす.ただし組織の挫滅に注意する(SCLCなど).
\item 普通の細胞診は\textbf{95\%エタノール}を用いて湿固定し,Papanicolaou染色を行う.\textbf{乾燥}を避ける.
\item 細胞診検体を遺伝子検査に使う場合,擦過器具を\textbf{生食}または\textbf{リン酸緩衝食塩液(PBS)}で洗ったものを保管する.
\item ROSEを行う場合は,よく乾燥させて,\textbf{diff-Quik}や\textbf{CytoQuick}などの簡易法で染色する.May-Giemsa染色と同様の見え方になる.
\item DNA,RNAの解析には,基本的には\textbf{凍結保存(液体窒素)}がbest.遺伝子解析を行うだけなら\textbf{10\%中性緩衝ホルマリン溶液}で\textbf{6-48時間}固定して,FFPE(ホルマリン固定パラフィン包埋)でもOK.ホルマリン固定を長時間行うとDNAの断片,修飾が起こる.
\item PD-L1はFFPE薄切後\textbf{6週}以内に検査する.
\item 生検体での検査が必要なのは\textbf{FCM},\textbf{染色体分析},\textbf{T細胞/B細胞受容体遺伝子の遺伝子解析}.
\item ホルマリン固定の検体でOKなのは,\textbf{FISH},\textbf{PCR},\textbf{NGS}.



\end{itemize}
